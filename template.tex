%***********************************************************************************
% ESTE FICHERO SE USA PARA UNIR VARIOS PDFs
%***********************************************************************************

\documentclass[a4paper,10pt]{article}
\usepackage[utf8x]{inputenc}

% Margins
\textheight 27.5cm 
\textwidth 18cm 
\oddsidemargin -1.5cm
\evensidemargin -1.5cm 
\topmargin -3cm

% Font
\usepackage{helvet}
% \fontfamily{phv}
% \selectfont
\renewcommand{\familydefault}{\sfdefault}

% Table format
\usepackage{array}
\setlength\extrarowheight{5pt}

% Logo
\usepackage{graphicx}
%\usepackage{fancyhdr}
\pagestyle{empty}

\begin{document}

\begin{center}
    \includegraphics[width=6cm]{Ciencias_UAM.pdf}
    %\vspace*{2cm}

    \rule{\textwidth}{0.2mm}
    \vspace*{0.2cm}
    
    \Large 
	\textbf{PROPUESTA DE PROYECTO FIN DE GRADO EN QUÍMICA (ID: _XXX_)}
\end{center}

\textbf{DATOS DEL TUTOR DEL PROYECTO Y DE LA ENTIDAD OFERTANTE}
\vspace*{0.2cm}

\begin{tabular}{|p{1.0\textwidth}|}
\hline
Nombre del tutor: _NAME_ \\\hline
Categoría profesional: _CAT_ \\\hline
Departamento, organismo al que pertenece: _DPTO_ \\\hline
e-mail: _EMAIL_ \\\hline
\end{tabular}
\vspace*{0.6cm}


\textbf{DATOS DEL PROYECTO}
\vspace*{0.2cm}

\begin{tabular}{|p{1.0\textwidth}|}
\hline
Título: _TITULO_ \\\hline
Título Inglés: _TITLE_ \\\hline
Tipo de trabajo: _TIPO_ \\\hline
Área de conocimiento en la que se enmarca: _AREA_ \\\hline
Breve descripción del tema, objetivos y tareas a realizar.\\
_DESCRIP_
\\\hline
Capacidades y conocimientos recomendados que el estudiante debería tener para llevar a cabo este proyecto.\\
_CAPAC_
\\\hline
Otra información relevante para ayudar a los estudiantes en su elección del trabajo fin de grado.\\
_INFO_\\\hline
Dirección del lugar donde se desarrollará el proyecto.\\
_DIR_\\\hline
\end{tabular}


\end{document}
